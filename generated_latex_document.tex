\documentclass[12pt]{article}
\usepackage{cite}
\usepackage{amsmath}
\usepackage{titling}

\title{IY3501: Question and Answer Lecture}
\author{Joe Reddington}
\date{}

\begin{document}

%Header written
\maketitle

\section*{Introduction}

Traditional lectures often follow a one-way flow of information, where the instructor presents material and students passively absorb the content. 
While this approach can be useful, it is not effective method for fostering deep understanding and repeatable skills. In contrast, a Q\&A-based lecture flips this dynamic and enables students to take an active role in their learning process.\cite{reidsema2017flipped} 

In a Q\&A-based lecture, the session is driven by the questions of the students, allowing the lecture to be more adaptive and responsive to their needs. This type of interactive learning environment has shown to be more engaging and effective, as students not only receive answers to their questions but also participate in the exploration of the topic alongside the instructor.\cite{ZhengLanqin2020TEot} 

By shifting the focus from passive listening to active participation, Q\&A-based lectures make learning more interactive, personalized, and effective. The effectiveness of such interactive models has been supported by studies that show improved learning outcomes and motivation in environments where active learning approaches, such as flipped classrooms, are used.\cite{ZhengLanqin2020TEot}

\section{Example Questions} 
Here are some randomly generated questions for you to try: 

%Intro written
\begin{itemize}
  \item Can I use an AI to write a risk assessment?
  \item Do all companies have to have ISO 27001 certification?
  \item How much do you trust your staff when they tell you about risks?
  \item Can you talk more about why some companies have NIST certification rather than ISO27001?
  \item If some of the ISO standards are law, then why do people have to pay for them?
  \item You say that there are lots of possible risks but ISO 27002 says there are only 34 possible controls?
  \item Are things like a hostile takeover a risk?
  \item I've read that ISO27001 requires you to have internal audits - can you tell us more?
  \item How can you evaluate the effectiveness of the ISMS if there haven't been any breaches?
  \item You showed us an email that wasn't spam: how do you check in practice?
  \item Why don't we start from a list of exploits when doing cyber risk assessment?
\end{itemize}

%Questions written
%Insert the questions here. 

\bibliographystyle{plain}
\bibliography{qanda}



%outro written
\maketitle

\section*{Introduction}

Traditional lectures often follow a one-way flow of information, where the instructor presents material and students passively absorb the content. 
While this approach can be useful, it is not effective method for fostering deep understanding and repeatable skills. In contrast, a Q\&A-based lecture flips this dynamic and enables students to take an active role in their learning process.\cite{reidsema2017flipped} 

In a Q\&A-based lecture, the session is driven by the questions of the students, allowing the lecture to be more adaptive and responsive to their needs. This type of interactive learning environment has shown to be more engaging and effective, as students not only receive answers to their questions but also participate in the exploration of the topic alongside the instructor.\cite{ZhengLanqin2020TEot} 

By shifting the focus from passive listening to active participation, Q\&A-based lectures make learning more interactive, personalized, and effective. The effectiveness of such interactive models has been supported by studies that show improved learning outcomes and motivation in environments where active learning approaches, such as flipped classrooms, are used.\cite{ZhengLanqin2020TEot}

\section{Example Questions} 
Here are some randomly generated questions for you to try: 

%Intro written
\begin{itemize}
  \item It seems like being risk averse might cause little problems, but being risk averse might cause big problems: so isn't it best to be a little risk averse?
  \item Aren't people incentivized to rate all risks as `low' because then they do less work?
  \item What sort of controls am I meant to put in for a Nation State attack?
  \item What do you do when something happens that wasn't in the risk assessment?
  \item How much do you trust your staff when they tell you about risks?
  \item Do all companies have to have ISO 27001 certification?
  \item How can I get different areas of the company to agree on how impact should be judged?
  \item If you follow GDPR are you automatically following ISO27001 or vice versa?
  \item If your control for one risk is `make backups' do you then need another risk for `backups get leaked'?
  \item How can you find out if another company is ISO 27001 certified?
  \item Is penetration testing required for ISO 27001?
\end{itemize}

%Questions written
%Insert the questions here. 

\bibliographystyle{plain}
\bibliography{qanda}



%outro written
\maketitle

\section*{Introduction}

Traditional lectures often follow a one-way flow of information, where the instructor presents material and students passively absorb the content. 
While this approach can be useful, it is not effective method for fostering deep understanding and repeatable skills. In contrast, a Q\&A-based lecture flips this dynamic and enables students to take an active role in their learning process.\cite{reidsema2017flipped} 

In a Q\&A-based lecture, the session is driven by the questions of the students, allowing the lecture to be more adaptive and responsive to their needs. This type of interactive learning environment has shown to be more engaging and effective, as students not only receive answers to their questions but also participate in the exploration of the topic alongside the instructor.\cite{ZhengLanqin2020TEot} 

By shifting the focus from passive listening to active participation, Q\&A-based lectures make learning more interactive, personalized, and effective. The effectiveness of such interactive models has been supported by studies that show improved learning outcomes and motivation in environments where active learning approaches, such as flipped classrooms, are used.\cite{ZhengLanqin2020TEot}

\section{Example Questions} 
Here are some randomly generated questions for you to try: 

%Intro written
\begin{itemize}
  \item Are insider attacks less detectable than outside ones?
  \item Why do cyber security people have to work in the office?
  \item How do you control risk culture at a company?
  \item Do you need to know if your staff have criminal records for a risk assessment and is that a GDPR issue?
  \item How can I get different areas of the company to agree on how impact should be judged?
  \item Why do the lecture notes talk about Treat and Terminate as responses to risk when the reading talks about a different four options?
  \item Who in a company can say `ignore that rule'?
  \item I hear that when you rate a risk as `high' in a company you get yelled at?
  \item Can we have more guest speakers?
  \item You showed us an email that wasn't spam: how do you check in practice?
  \item If you follow GDPR are you automatically following ISO27001 or vice versa?
\end{itemize}

%Questions written
%Insert the questions here. 

\bibliographystyle{plain}
\bibliography{qanda}



%outro written
\maketitle

\section*{Introduction}

Traditional lectures often follow a one-way flow of information, where the instructor presents material and students passively absorb the content. 
While this approach can be useful, it is not effective method for fostering deep understanding and repeatable skills. In contrast, a Q\&A-based lecture flips this dynamic and enables students to take an active role in their learning process.\cite{reidsema2017flipped} 

In a Q\&A-based lecture, the session is driven by the questions of the students, allowing the lecture to be more adaptive and responsive to their needs. This type of interactive learning environment has shown to be more engaging and effective, as students not only receive answers to their questions but also participate in the exploration of the topic alongside the instructor.\cite{ZhengLanqin2020TEot} 

By shifting the focus from passive listening to active participation, Q\&A-based lectures make learning more interactive, personalized, and effective. The effectiveness of such interactive models has been supported by studies that show improved learning outcomes and motivation in environments where active learning approaches, such as flipped classrooms, are used.\cite{ZhengLanqin2020TEot}

\section{Example Questions} 
Here are some randomly generated questions for you to try: 

%Intro written
\begin{itemize}
  \item Should risk assessments be done in-house or outsourced to specialists?
  \item How can you find out if another company is ISO 27001 certified?
  \item What do you do when something happens that wasn't in the risk assessment?
  \item What happens if you make a risk assessment for a company but the CEO hasn't told you their secret plans?
  \item How do companies work out how much budget to spend dealing with risks?
  \item Do you need to know if your staff have criminal records for a risk assessment and is that a GDPR issue?
  \item For things like death and data breaches, surely the only acceptable risk is zero?
  \item Are there any risks that you can just always put on the list?
  \item Why a company gets attacked, how can they tell what sort of attacker it is?
  \item Are things like a hostile takeover a risk?
  \item Why is all the wording in ISO27001 so old?
\end{itemize}

%Questions written
%Insert the questions here. 

\bibliographystyle{plain}
\bibliography{qanda}



%outro written
\maketitle

\section*{Introduction}

Traditional lectures often follow a one-way flow of information, where the instructor presents material and students passively absorb the content. 
While this approach can be useful, it is not effective method for fostering deep understanding and repeatable skills. In contrast, a Q\&A-based lecture flips this dynamic and enables students to take an active role in their learning process.\cite{reidsema2017flipped} 

In a Q\&A-based lecture, the session is driven by the questions of the students, allowing the lecture to be more adaptive and responsive to their needs. This type of interactive learning environment has shown to be more engaging and effective, as students not only receive answers to their questions but also participate in the exploration of the topic alongside the instructor.\cite{ZhengLanqin2020TEot} 

By shifting the focus from passive listening to active participation, Q\&A-based lectures make learning more interactive, personalized, and effective. The effectiveness of such interactive models has been supported by studies that show improved learning outcomes and motivation in environments where active learning approaches, such as flipped classrooms, are used.\cite{ZhengLanqin2020TEot}

\section{Example Questions} 
Here are some randomly generated questions for you to try: 

%Intro written
\begin{itemize}
  \item Are there any risks that you can just always put on the list?
  \item For things like death and data breaches, surely the only acceptable risk is zero?
  \item Should risk assessments be done in-house or outsourced to specialists?
  \item Can we have more guest speakers?
  \item Are there disadvantages to doing a Risk-based assessment of a company?
  \item Are things like a hostile takeover a risk?
  \item Are we going to talk about public relations for when there's a breach or just the technical side?
  \item What sort of controls am I meant to put in for a Nation State attack?
  \item How can you monitor ISO27001 stuff if everybody is home working?
  \item What's a Risk Register?
  \item If you have to pay to be given an ISO 27000 certificate then why would anyone ever get failed?
\end{itemize}

%Questions written
%Insert the questions here. 

\bibliographystyle{plain}
\bibliography{qanda}



%outro written
\maketitle

\section*{Introduction}

Traditional lectures often follow a one-way flow of information, where the instructor presents material and students passively absorb the content. 
While this approach can be useful, it is not effective method for fostering deep understanding and repeatable skills. In contrast, a Q\&A-based lecture flips this dynamic and enables students to take an active role in their learning process.\cite{reidsema2017flipped} 

In a Q\&A-based lecture, the session is driven by the questions of the students, allowing the lecture to be more adaptive and responsive to their needs. This type of interactive learning environment has shown to be more engaging and effective, as students not only receive answers to their questions but also participate in the exploration of the topic alongside the instructor.\cite{ZhengLanqin2020TEot} 

By shifting the focus from passive listening to active participation, Q\&A-based lectures make learning more interactive, personalized, and effective. The effectiveness of such interactive models has been supported by studies that show improved learning outcomes and motivation in environments where active learning approaches, such as flipped classrooms, are used.\cite{ZhengLanqin2020TEot}

\section{Example Questions} 
Here are some randomly generated questions for you to try: 

%Intro written
\begin{itemize}
  \item If you have to pay to be given an ISO 27000 certificate then why would anyone ever get failed?
  \item Can you talk more about why some companies have NIST certification rather than ISO27001?
  \item Do you sometimes write one risk assessment for an auditor and another for `real'?
  \item (after 2pm) can we have our break now?
  \item How much do you trust your staff when they tell you about risks?
  \item What do you do when something happens that wasn't in the risk assessment?
  \item Is Penetration Testing part of Risk Assessment?
  \item Why is all the wording in ISO27001 so old?
  \item Some Risk Assessments multiply likelihood and impact but some add them: which one is correct?
  \item How do you risk assess something like a pandemic?
  \item All the attackers you taught seem like they do really impactful things on the risk assessments, are there any low-impact attackers?
\end{itemize}

%Questions written
%Insert the questions here. 

\bibliographystyle{plain}
\bibliography{qanda}



%outro written
\maketitle

\section*{Introduction}

Traditional lectures often follow a one-way flow of information, where the instructor presents material and students passively absorb the content. 
While this approach can be useful, it is not effective method for fostering deep understanding and repeatable skills. In contrast, a Q\&A-based lecture flips this dynamic and enables students to take an active role in their learning process.\cite{reidsema2017flipped} 

In a Q\&A-based lecture, the session is driven by the questions of the students, allowing the lecture to be more adaptive and responsive to their needs. This type of interactive learning environment has shown to be more engaging and effective, as students not only receive answers to their questions but also participate in the exploration of the topic alongside the instructor.\cite{ZhengLanqin2020TEot} 

By shifting the focus from passive listening to active participation, Q\&A-based lectures make learning more interactive, personalized, and effective. The effectiveness of such interactive models has been supported by studies that show improved learning outcomes and motivation in environments where active learning approaches, such as flipped classrooms, are used.\cite{ZhengLanqin2020TEot}

\section{Example Questions} 
Here are some randomly generated questions for you to try: 

%Intro written
\begin{itemize}
  \item I hear that when you rate a risk as `high' in a company you get yelled at?
  \item How can you monitor ISO27001 stuff if everybody is home working?
  \item Is Penetration Testing part of Risk Assessment?
  \item Why is all the wording in ISO27001 so old?
  \item How much of a security professional's time ends up being spent on risks and standards?
  \item How do I persuade people at my company to act on the risk assessment?
  \item For things like death and data breaches, surely the only acceptable risk is zero?
  \item Are we going to talk about public relations for when there's a breach or just the technical side?
  \item If humans are so terrible at risk, then should we bother with the risk assessment at all?
  \item Is a risk assessment confidential?
  \item What sort of controls am I meant to put in for a Nation State attack?
\end{itemize}

%Questions written
%Insert the questions here. 

\bibliographystyle{plain}
\bibliography{qanda}



%outro written
\maketitle

\section*{Introduction}

Traditional lectures often follow a one-way flow of information, where the instructor presents material and students passively absorb the content. 
While this approach can be useful, it is not effective method for fostering deep understanding and repeatable skills. In contrast, a Q\&A-based lecture flips this dynamic and enables students to take an active role in their learning process.\cite{reidsema2017flipped} 

In a Q\&A-based lecture, the session is driven by the questions of the students, allowing the lecture to be more adaptive and responsive to their needs. This type of interactive learning environment has shown to be more engaging and effective, as students not only receive answers to their questions but also participate in the exploration of the topic alongside the instructor.\cite{ZhengLanqin2020TEot} 

By shifting the focus from passive listening to active participation, Q\&A-based lectures make learning more interactive, personalized, and effective. The effectiveness of such interactive models has been supported by studies that show improved learning outcomes and motivation in environments where active learning approaches, such as flipped classrooms, are used.\cite{ZhengLanqin2020TEot}

\section{Example Questions} 
Here are some randomly generated questions for you to try: 

%Intro written
\begin{itemize}
  \item If you follow GDPR are you automatically following ISO27001 or vice versa?
  \item Do you need to know if your staff have criminal records for a risk assessment and is that a GDPR issue?
  \item Is Penetration Testing part of Risk Assessment?
  \item Why is all the wording in ISO27001 so old?
  \item Don't most companies have to worry about the same set of risks?
  \item What do you do when something happens that wasn't in the risk assessment?
  \item Can we have more guest speakers?
  \item Do you include phones belonging to staff in a risk assessment?
  \item How can you monitor ISO27001 stuff if everybody is home working?
  \item It seems like being risk averse might cause little problems, but being risk averse might cause big problems: so isn't it best to be a little risk averse?
  \item Are things like a hostile takeover a risk?
\end{itemize}

%Questions written
%Insert the questions here. 

\bibliographystyle{plain}
\bibliography{qanda}



%outro written
\maketitle

\section*{Introduction}

Traditional lectures often follow a one-way flow of information, where the instructor presents material and students passively absorb the content. 
While this approach can be useful, it is not effective method for fostering deep understanding and repeatable skills. In contrast, a Q\&A-based lecture flips this dynamic and enables students to take an active role in their learning process.\cite{reidsema2017flipped} 

In a Q\&A-based lecture, the session is driven by the questions of the students, allowing the lecture to be more adaptive and responsive to their needs. This type of interactive learning environment has shown to be more engaging and effective, as students not only receive answers to their questions but also participate in the exploration of the topic alongside the instructor.\cite{ZhengLanqin2020TEot} 

By shifting the focus from passive listening to active participation, Q\&A-based lectures make learning more interactive, personalized, and effective. The effectiveness of such interactive models has been supported by studies that show improved learning outcomes and motivation in environments where active learning approaches, such as flipped classrooms, are used.\cite{ZhengLanqin2020TEot}

\section{Example Questions} 
Here are some randomly generated questions for you to try: 

%Intro written
\begin{itemize}
  \item How much do you trust your staff when they tell you about risks?
  \item It seems like being risk averse might cause little problems, but being risk averse might cause big problems: so isn't it best to be a little risk averse?
  \item Can we see a Royal Holloway Risk assessment?
  \item If some of the ISO standards are law, then why do people have to pay for them?
  \item Are insider attacks less detectable than outside ones?
  \item Are there disadvantages to doing a Risk-based assessment of a company?
  \item Can we have more guest speakers?
  \item You say that there are lots of possible risks but ISO 27002 says there are only 34 possible controls?
  \item You taught that one of the attackers was `cyber criminals' but aren't all the attackers some kind of cyber criminal?
  \item Have you caused a data breach?
  \item How can we keep all the records we need for accountability while still being okay with GDPR?
\end{itemize}

%Questions written
%Insert the questions here. 

\bibliographystyle{plain}
\bibliography{qanda}



%outro written
\maketitle

\section*{Introduction}

Traditional lectures often follow a one-way flow of information, where the instructor presents material and students passively absorb the content. 
While this approach can be useful, it is not effective method for fostering deep understanding and repeatable skills. In contrast, a Q\&A-based lecture flips this dynamic and enables students to take an active role in their learning process.\cite{reidsema2017flipped} 

In a Q\&A-based lecture, the session is driven by the questions of the students, allowing the lecture to be more adaptive and responsive to their needs. This type of interactive learning environment has shown to be more engaging and effective, as students not only receive answers to their questions but also participate in the exploration of the topic alongside the instructor.\cite{ZhengLanqin2020TEot} 

By shifting the focus from passive listening to active participation, Q\&A-based lectures make learning more interactive, personalized, and effective. The effectiveness of such interactive models has been supported by studies that show improved learning outcomes and motivation in environments where active learning approaches, such as flipped classrooms, are used.\cite{ZhengLanqin2020TEot}

\section{Example Questions} 
Here are some randomly generated questions for you to try: 

%Intro written
\begin{itemize}
  \item It seems like being risk averse might cause little problems, but being risk averse might cause big problems: so isn't it best to be a little risk averse?
  \item Do you sometimes write one risk assessment for an auditor and another for `real'?
  \item How can you monitor ISO27001 stuff if everybody is home working?
  \item If you follow GDPR are you automatically following ISO27001 or vice versa?
  \item If you have to pay to be given an ISO 27000 certificate then why would anyone ever get failed?
  \item When it's a high level risk assessment, would you do the impact on the share price rather than overall cost?
  \item I hear that when you rate a risk as `high' in a company you get yelled at?
  \item How do startups do risk assessment when they are scaling so quickly?
  \item Why do the slides say that giving risk a number isn't objective when the reading says it is?
  \item Can only part of a company (a department or something) get certified to a standard?
  \item What are the big things that can go wrong with a risk assessment process?
\end{itemize}

%Questions written
%Insert the questions here. 

\bibliographystyle{plain}
\bibliography{qanda}



%outro written
\end{document}


