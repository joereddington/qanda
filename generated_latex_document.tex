\documentclass[12pt]{article}
\usepackage{cite}
\usepackage{amsmath}
\usepackage{titling}

\title{IY3501: Question and Answer Lecture}
\author{Joe Reddington}
\date{}

\begin{document}

%Header written
\maketitle

\section*{Introduction}
We're going to do another question and answer session today. This time I'd like to focus on the second half of the course but I am happy to take questions from any section of the course, or from any of the recommended readings. 

If you are short of questions, then I suggest: 

\begin{itemize} 
\item looking through previous slides and seeing if there are any parts you don't feel like you `get';
\item looking through past papers and seeing if there are any questions you don't feel like you could answer;
\item asking some of the starter questions.
\end{itemize} 

You can ask questions by putting up your hand or with Teams.  

Particularly interesting questions will be added to the database and may be starter questions for next year.  

\section*{Starter Questions} 
Here are some randomly generated questions for you to try asking: 

%Intro written
\begin{itemize}
  \item How do you put a new security policy in place at a large organisation - can you just email and say `we're doing it this way now?'
  \item How do criminal record checks differ internationally?
  \item I read that the Risk Register should be confidential - doesn't that make it useless?
  \item How do corrective actions during incident management differ from risk treatments?
  \item How can data breaches alter public perceptions of organizational ethics?
  \item How does pseudonymization differ from anonymization under GDPR?
  \item Why do we separate into the five phases of incident management?
  \item Is `Risk Appetite' the same as 'Risk Culture'
  \item How does security management work for a computer games company?
  \item How can organizations use GDPR principles to build trust with customers?
  \item Have you ever been scammed?
\end{itemize}

%Questions written
%Insert the questions here. 

%\bibliographystyle{plain}
%\bibliography{qanda}



%outro written
\maketitle

\section*{Introduction}
We're going to do another question and answer session today. This time I'd like to focus on the second half of the course but I am happy to take questions from any section of the course, or from any of the recommended readings. 

If you are short of questions, then I suggest: 

\begin{itemize} 
\item looking through previous slides and seeing if there are any parts you don't feel like you `get';
\item looking through past papers and seeing if there are any questions you don't feel like you could answer;
\item asking some of the starter questions.
\end{itemize} 

You can ask questions by putting up your hand or with Teams.  

Particularly interesting questions will be added to the database and may be starter questions for next year.  

\section*{Starter Questions} 
Here are some randomly generated questions for you to try asking: 

%Intro written
\begin{itemize}
  \item How does security management work for a computer games company?
  \item Can I pass an audit by putting nothing on the statement of applicability?
  \item Is a statement of applicability public?
  \item Tiger kidnappings seem really interesting - do you have any thoughts?
  \item What's the most important job of an Incident Response Team (IRT)?
  \item In lots of the case studies a company calls in external consultants - is there a rule for when you do that?
  \item What is the role of risk assessment in disaster recovery planning?
  \item At what point in a security incident should you tell the police??
  \item I heard some guy say that you could use NIST controls for an ISO27001 audit - is that okay?
  \item What are the implications of the right to be forgotten in modern business?
  \item How do systemic biases affect diversity in cybersecurity careers?
\end{itemize}

%Questions written
%Insert the questions here. 

%\bibliographystyle{plain}
%\bibliography{qanda}



%outro written
\maketitle

\section*{Introduction}
We're going to do another question and answer session today. This time I'd like to focus on the second half of the course but I am happy to take questions from any section of the course, or from any of the recommended readings. 

If you are short of questions, then I suggest: 

\begin{itemize} 
\item looking through previous slides and seeing if there are any parts you don't feel like you `get';
\item looking through past papers and seeing if there are any questions you don't feel like you could answer;
\item asking some of the starter questions.
\end{itemize} 

You can ask questions by putting up your hand or with Teams.  

Particularly interesting questions will be added to the database and may be starter questions for next year.  

\section*{Starter Questions} 
Here are some randomly generated questions for you to try asking: 

%Intro written
\begin{itemize}
  \item I heard some guy say that you could use NIST controls for an ISO27001 audit - is that okay?
  \item Why do we separate into the five phases of incident management?
  \item Can I pass an audit by putting nothing on the statement of applicability?
  \item Given that equipment is really cheap compared to the cost of employees, why is the return of assets important when an employee leaves?
  \item Surely many policies are irrelevant if people have work email on their phones?
  \item How do systemic biases affect diversity in cybersecurity careers?
  \item What's the difference between Risk Management and Enterprise Risk Management?
  \item At what point in a security incident should you tell the police??
  \item How do organizations balance trust with thorough screening processes?
  \item I read that the Risk Register should be confidential - doesn't that make it useless?
  \item Is `Risk Appetite' the same as 'Risk Culture'
\end{itemize}

%Questions written
%Insert the questions here. 

%\bibliographystyle{plain}
%\bibliography{qanda}



%outro written
\maketitle

\section*{Introduction}
We're going to do another question and answer session today. This time I'd like to focus on the second half of the course but I am happy to take questions from any section of the course, or from any of the recommended readings. 

If you are short of questions, then I suggest: 

\begin{itemize} 
\item looking through previous slides and seeing if there are any parts you don't feel like you `get';
\item looking through past papers and seeing if there are any questions you don't feel like you could answer;
\item asking some of the starter questions.
\end{itemize} 

You can ask questions by putting up your hand or with Teams.  

Particularly interesting questions will be added to the database and may be starter questions for next year.  

\section*{Starter Questions} 
Here are some randomly generated questions for you to try asking: 

%Intro written
\begin{itemize}
  \item What's the difference between Risk Management and Enterprise Risk Management?
  \item I read that the Risk Register should be confidential - doesn't that make it useless?
  \item How do organizations determine the scope of their disaster recovery plans?
  \item Can I pass an audit by putting nothing on the statement of applicability?
  \item What is the difference between data controllers and data processors?
  \item What's the most important job of an Incident Response Team (IRT)?
  \item Surely many policies are irrelevant if people have work email on their phones?
  \item Have you ever been scammed?
  \item How can data breaches alter public perceptions of organizational ethics?
  \item Is a statement of applicability public?
  \item Should we talk about stakeholder communication in incident management? It seems important...
\end{itemize}

%Questions written
%Insert the questions here. 

%\bibliographystyle{plain}
%\bibliography{qanda}



%outro written
\maketitle

\section*{Introduction}
We're going to do another question and answer session today. This time I'd like to focus on the second half of the course but I am happy to take questions from any section of the course, or from any of the recommended readings. 

If you are short of questions, then I suggest: 

\begin{itemize} 
\item looking through previous slides and seeing if there are any parts you don't feel like you `get';
\item looking through past papers and seeing if there are any questions you don't feel like you could answer;
\item asking some of the starter questions.
\end{itemize} 

You can ask questions by putting up your hand or with Teams.  

Particularly interesting questions will be added to the database and may be starter questions for next year.  

\section*{Starter Questions} 
Here are some randomly generated questions for you to try asking: 

%Intro written
\begin{itemize}
  \item How does security management work for a military unit?
  \item How do criminal record checks differ internationally?
  \item Say that I write a security policy and an employee ignores it and there is a breach - can I be blamed?
  \item In lots of the case studies a company calls in external consultants - is there a rule for when you do that?
  \item How does security management work for a computer games company?
  \item Is a statement of applicability public?
  \item Should we talk about stakeholder communication in incident management? It seems important...
  \item Why is training staff to recognize incidents a top priority?
  \item What are the implications of the right to be forgotten in modern business?
  \item Why do we separate into the five phases of incident management?
  \item How can organizations use GDPR principles to build trust with customers?
\end{itemize}

%Questions written
%Insert the questions here. 

%\bibliographystyle{plain}
%\bibliography{qanda}



%outro written
\maketitle

\section*{Introduction}
We're going to do another question and answer session today. This time I'd like to focus on the second half of the course but I am happy to take questions from any section of the course, or from any of the recommended readings. 

If you are short of questions, then I suggest: 

\begin{itemize} 
\item looking through previous slides and seeing if there are any parts you don't feel like you `get';
\item looking through past papers and seeing if there are any questions you don't feel like you could answer;
\item asking some of the starter questions.
\end{itemize} 

You can ask questions by putting up your hand or with Teams.  

Particularly interesting questions will be added to the database and may be starter questions for next year.  

\section*{Starter Questions} 
Here are some randomly generated questions for you to try asking: 

%Intro written
\begin{itemize}
  \item Can I pass an audit by putting nothing on the statement of applicability?
  \item How do organizations balance trust with thorough screening processes?
  \item How does security management work for a computer games company?
  \item Is a statement of applicability public?
  \item How do you put a new security policy in place at a large organisation - can you just email and say `we're doing it this way now?'
  \item How do phishing simulations contribute to employee training?
  \item How do systemic biases affect diversity in cybersecurity careers?
  \item What's the most important job of an Incident Response Team (IRT)?
  \item Why is the 'sliding scale' of spam relevance a key concern in marketing ethics?
  \item How does pseudonymization differ from anonymization under GDPR?
  \item Why do we separate into the five phases of incident management?
\end{itemize}

%Questions written
%Insert the questions here. 

%\bibliographystyle{plain}
%\bibliography{qanda}



%outro written
\maketitle

\section*{Introduction}
We're going to do another question and answer session today. This time I'd like to focus on the second half of the course but I am happy to take questions from any section of the course, or from any of the recommended readings. 

If you are short of questions, then I suggest: 

\begin{itemize} 
\item looking through previous slides and seeing if there are any parts you don't feel like you `get';
\item looking through past papers and seeing if there are any questions you don't feel like you could answer;
\item asking some of the starter questions.
\end{itemize} 

You can ask questions by putting up your hand or with Teams.  

Particularly interesting questions will be added to the database and may be starter questions for next year.  

\section*{Starter Questions} 
Here are some randomly generated questions for you to try asking: 

%Intro written
\begin{itemize}
  \item Say that I write a security policy and an employee ignores it and there is a breach - can I be blamed?
  \item How does pseudonymization differ from anonymization under GDPR?
  \item I read that the Risk Register should be confidential - doesn't that make it useless?
  \item Why is the 'sliding scale' of spam relevance a key concern in marketing ethics?
  \item I heard some guy say that you could use NIST controls for an ISO27001 audit - is that okay?
  \item In lots of the case studies a company calls in external consultants - is there a rule for when you do that?
  \item What's the difference between Risk Management and Enterprise Risk Management?
  \item Is a statement of applicability public?
  \item Why do we separate into the five phases of incident management?
  \item Can I pass an audit by putting nothing on the statement of applicability?
  \item How does security management work for a military unit?
\end{itemize}

%Questions written
%Insert the questions here. 

%\bibliographystyle{plain}
%\bibliography{qanda}



%outro written
\maketitle

\section*{Introduction}
We're going to do another question and answer session today. This time I'd like to focus on the second half of the course but I am happy to take questions from any section of the course, or from any of the recommended readings. 

If you are short of questions, then I suggest: 

\begin{itemize} 
\item looking through previous slides and seeing if there are any parts you don't feel like you `get';
\item looking through past papers and seeing if there are any questions you don't feel like you could answer;
\item asking some of the starter questions.
\end{itemize} 

You can ask questions by putting up your hand or with Teams.  

Particularly interesting questions will be added to the database and may be starter questions for next year.  

\section*{Starter Questions} 
Here are some randomly generated questions for you to try asking: 

%Intro written
\begin{itemize}
  \item Why is the 'sliding scale' of spam relevance a key concern in marketing ethics?
  \item How does security management work for a military unit?
  \item How does security management work for a computer games company?
  \item Should we talk about stakeholder communication in incident management? It seems important...
  \item It feels like all of the ransomware stuff is solved by just 'having a backup' do so many companies really just not back stuff up?
  \item Say that I write a security policy and an employee ignores it and there is a breach - can I be blamed?
  \item I read that the Risk Register should be confidential - doesn't that make it useless?
  \item Is `Risk Appetite' the same as 'Risk Culture'
  \item What's the difference between Risk Management and Enterprise Risk Management?
  \item How do you put a new security policy in place at a large organisation - can you just email and say `we're doing it this way now?'
  \item How can data breaches alter public perceptions of organizational ethics?
\end{itemize}

%Questions written
%Insert the questions here. 

%\bibliographystyle{plain}
%\bibliography{qanda}



%outro written
\maketitle

\section*{Introduction}
We're going to do another question and answer session today. This time I'd like to focus on the second half of the course but I am happy to take questions from any section of the course, or from any of the recommended readings. 

If you are short of questions, then I suggest: 

\begin{itemize} 
\item looking through previous slides and seeing if there are any parts you don't feel like you `get';
\item looking through past papers and seeing if there are any questions you don't feel like you could answer;
\item asking some of the starter questions.
\end{itemize} 

You can ask questions by putting up your hand or with Teams.  

Particularly interesting questions will be added to the database and may be starter questions for next year.  

\section*{Starter Questions} 
Here are some randomly generated questions for you to try asking: 

%Intro written
\begin{itemize}
  \item Why might attackers exploit vulnerabilities during disaster recovery?
  \item Why is training staff to recognize incidents a top priority?
  \item I heard some guy say that you could use NIST controls for an ISO27001 audit - is that okay?
  \item Why is the 'sliding scale' of spam relevance a key concern in marketing ethics?
  \item How does security management work for a military unit?
  \item How do organizations balance trust with thorough screening processes?
  \item How can organizations use GDPR principles to build trust with customers?
  \item Is `Risk Appetite' the same as 'Risk Culture'
  \item How do criminal record checks differ internationally?
  \item Say that I write a security policy and an employee ignores it and there is a breach - can I be blamed?
  \item At what point in a security incident should you tell the police??
\end{itemize}

%Questions written
%Insert the questions here. 

%\bibliographystyle{plain}
%\bibliography{qanda}



%outro written
\maketitle

\section*{Introduction}
We're going to do another question and answer session today. This time I'd like to focus on the second half of the course but I am happy to take questions from any section of the course, or from any of the recommended readings. 

If you are short of questions, then I suggest: 

\begin{itemize} 
\item looking through previous slides and seeing if there are any parts you don't feel like you `get';
\item looking through past papers and seeing if there are any questions you don't feel like you could answer;
\item asking some of the starter questions.
\end{itemize} 

You can ask questions by putting up your hand or with Teams.  

Particularly interesting questions will be added to the database and may be starter questions for next year.  

\section*{Starter Questions} 
Here are some randomly generated questions for you to try asking: 

%Intro written
\begin{itemize}
  \item How does security management work for a military unit?
  \item What's the most important job of an Incident Response Team (IRT)?
  \item What is the role of risk assessment in disaster recovery planning?
  \item How do organizations determine the scope of their disaster recovery plans?
  \item What are the implications of the right to be forgotten in modern business?
  \item Surely many policies are irrelevant if people have work email on their phones?
  \item It feels like all of the ransomware stuff is solved by just 'having a backup' do so many companies really just not back stuff up?
  \item How does pseudonymization differ from anonymization under GDPR?
  \item Is `Risk Appetite' the same as 'Risk Culture'
  \item What is the difference between data controllers and data processors?
  \item How does security management work for a computer games company?
\end{itemize}

%Questions written
%Insert the questions here. 

%\bibliographystyle{plain}
%\bibliography{qanda}



%outro written
\maketitle

\section*{Introduction}
We're going to do another question and answer session today. This time I'd like to focus on the second half of the course but I am happy to take questions from any section of the course, or from any of the recommended readings. 

If you are short of questions, then I suggest: 

\begin{itemize} 
\item looking through previous slides and seeing if there are any parts you don't feel like you `get';
\item looking through past papers and seeing if there are any questions you don't feel like you could answer;
\item asking some of the starter questions.
\end{itemize} 

You can ask questions by putting up your hand or with Teams.  

Particularly interesting questions will be added to the database and may be starter questions for next year.  

\section*{Starter Questions} 
Here are some randomly generated questions for you to try asking: 

%Intro written
\begin{itemize}
  \item How do you put a new security policy in place at a large organisation - can you just email and say `we're doing it this way now?'
  \item I heard some guy say that you could use NIST controls for an ISO27001 audit - is that okay?
  \item How can organizations use GDPR principles to build trust with customers?
  \item How do corrective actions during incident management differ from risk treatments?
  \item Can I pass an audit by putting nothing on the statement of applicability?
  \item Is `Risk Appetite' the same as 'Risk Culture'
  \item Why do we separate into the five phases of incident management?
  \item What's the most important job of an Incident Response Team (IRT)?
  \item What is the difference between data controllers and data processors?
  \item It feels like all of the ransomware stuff is solved by just 'having a backup' do so many companies really just not back stuff up?
  \item What is the role of risk assessment in disaster recovery planning?
\end{itemize}

%Questions written
%Insert the questions here. 

%\bibliographystyle{plain}
%\bibliography{qanda}



%outro written
\maketitle

\section*{Introduction}
We're going to do another question and answer session today. This time I'd like to focus on the second half of the course but I am happy to take questions from any section of the course, or from any of the recommended readings. 

If you are short of questions, then I suggest: 

\begin{itemize} 
\item looking through previous slides and seeing if there are any parts you don't feel like you `get';
\item looking through past papers and seeing if there are any questions you don't feel like you could answer;
\item asking some of the starter questions.
\end{itemize} 

You can ask questions by putting up your hand or with Teams.  

Particularly interesting questions will be added to the database and may be starter questions for next year.  

\section*{Starter Questions} 
Here are some randomly generated questions for you to try asking: 

%Intro written
\begin{itemize}
  \item Should we talk about stakeholder communication in incident management? It seems important...
  \item Tiger kidnappings seem really interesting - do you have any thoughts?
  \item What are the implications of the right to be forgotten in modern business?
  \item How do systemic biases affect diversity in cybersecurity careers?
  \item How can organizations use GDPR principles to build trust with customers?
  \item What is the difference between data controllers and data processors?
  \item Why is training staff to recognize incidents a top priority?
  \item Why do we separate into the five phases of incident management?
  \item Given that equipment is really cheap compared to the cost of employees, why is the return of assets important when an employee leaves?
  \item Is `Risk Appetite' the same as 'Risk Culture'
  \item Surely many policies are irrelevant if people have work email on their phones?
\end{itemize}

%Questions written
%Insert the questions here. 

%\bibliographystyle{plain}
%\bibliography{qanda}



%outro written
\maketitle

\section*{Introduction}
We're going to do another question and answer session today. This time I'd like to focus on the second half of the course but I am happy to take questions from any section of the course, or from any of the recommended readings. 

If you are short of questions, then I suggest: 

\begin{itemize} 
\item looking through previous slides and seeing if there are any parts you don't feel like you `get';
\item looking through past papers and seeing if there are any questions you don't feel like you could answer;
\item asking some of the starter questions.
\end{itemize} 

You can ask questions by putting up your hand or with Teams.  

Particularly interesting questions will be added to the database and may be starter questions for next year.  

\section*{Starter Questions} 
Here are some randomly generated questions for you to try asking: 

%Intro written
\begin{itemize}
  \item In lots of the case studies a company calls in external consultants - is there a rule for when you do that?
  \item I read that the Risk Register should be confidential - doesn't that make it useless?
  \item What is the difference between data controllers and data processors?
  \item Why do we separate into the five phases of incident management?
  \item How do organizations determine the scope of their disaster recovery plans?
  \item How does security management work for a computer games company?
  \item Can I pass an audit by putting nothing on the statement of applicability?
  \item Have you ever been scammed?
  \item Tiger kidnappings seem really interesting - do you have any thoughts?
  \item How does pseudonymization differ from anonymization under GDPR?
  \item How do systemic biases affect diversity in cybersecurity careers?
\end{itemize}

%Questions written
%Insert the questions here. 

%\bibliographystyle{plain}
%\bibliography{qanda}



%outro written
\maketitle

\section*{Introduction}
We're going to do another question and answer session today. This time I'd like to focus on the second half of the course but I am happy to take questions from any section of the course, or from any of the recommended readings. 

If you are short of questions, then I suggest: 

\begin{itemize} 
\item looking through previous slides and seeing if there are any parts you don't feel like you `get';
\item looking through past papers and seeing if there are any questions you don't feel like you could answer;
\item asking some of the starter questions.
\end{itemize} 

You can ask questions by putting up your hand or with Teams.  

Particularly interesting questions will be added to the database and may be starter questions for next year.  

\section*{Starter Questions} 
Here are some randomly generated questions for you to try asking: 

%Intro written
\begin{itemize}
  \item How do phishing simulations contribute to employee training?
  \item Why is the 'sliding scale' of spam relevance a key concern in marketing ethics?
  \item I heard some guy say that you could use NIST controls for an ISO27001 audit - is that okay?
  \item Is a statement of applicability public?
  \item What's the most important job of an Incident Response Team (IRT)?
  \item Say that I write a security policy and an employee ignores it and there is a breach - can I be blamed?
  \item How can organizations use GDPR principles to build trust with customers?
  \item How do organizations determine the scope of their disaster recovery plans?
  \item How does security management work for a military unit?
  \item What is the difference between data controllers and data processors?
  \item Given that equipment is really cheap compared to the cost of employees, why is the return of assets important when an employee leaves?
\end{itemize}

%Questions written
%Insert the questions here. 

%\bibliographystyle{plain}
%\bibliography{qanda}



%outro written
\maketitle

\section*{Introduction}
We're going to do another question and answer session today. This time I'd like to focus on the second half of the course but I am happy to take questions from any section of the course, or from any of the recommended readings. 

If you are short of questions, then I suggest: 

\begin{itemize} 
\item looking through previous slides and seeing if there are any parts you don't feel like you `get';
\item looking through past papers and seeing if there are any questions you don't feel like you could answer;
\item asking some of the starter questions.
\end{itemize} 

You can ask questions by putting up your hand or with Teams.  

Particularly interesting questions will be added to the database and may be starter questions for next year.  

\section*{Starter Questions} 
Here are some randomly generated questions for you to try asking: 

%Intro written
\begin{itemize}
  \item What are the implications of the right to be forgotten in modern business?
  \item Why might attackers exploit vulnerabilities during disaster recovery?
  \item How do organizations balance trust with thorough screening processes?
  \item How can data breaches alter public perceptions of organizational ethics?
  \item How do corrective actions during incident management differ from risk treatments?
  \item How does security management work for a military unit?
  \item What is the role of risk assessment in disaster recovery planning?
  \item In lots of the case studies a company calls in external consultants - is there a rule for when you do that?
  \item Can I pass an audit by putting nothing on the statement of applicability?
  \item What's the most important job of an Incident Response Team (IRT)?
  \item How do phishing simulations contribute to employee training?
\end{itemize}

%Questions written
%Insert the questions here. 

%\bibliographystyle{plain}
%\bibliography{qanda}



%outro written
\maketitle

\section*{Introduction}
We're going to do another question and answer session today. This time I'd like to focus on the second half of the course but I am happy to take questions from any section of the course, or from any of the recommended readings. 

If you are short of questions, then I suggest: 

\begin{itemize} 
\item looking through previous slides and seeing if there are any parts you don't feel like you `get';
\item looking through past papers and seeing if there are any questions you don't feel like you could answer;
\item asking some of the starter questions.
\end{itemize} 

You can ask questions by putting up your hand or with Teams.  

Particularly interesting questions will be added to the database and may be starter questions for next year.  

\section*{Starter Questions} 
Here are some randomly generated questions for you to try asking: 

%Intro written
\begin{itemize}
  \item What is the difference between data controllers and data processors?
  \item How do corrective actions during incident management differ from risk treatments?
  \item How does security management work for a military unit?
  \item How do criminal record checks differ internationally?
  \item Should we talk about stakeholder communication in incident management? It seems important...
  \item Are there any companies that shouldn't have an ISMS?
  \item I heard some guy say that you could use NIST controls for an ISO27001 audit - is that okay?
  \item What is the role of risk assessment in disaster recovery planning?
  \item Tiger kidnappings seem really interesting - do you have any thoughts?
  \item How do you put a new security policy in place at a large organisation - can you just email and say `we're doing it this way now?'
  \item How do phishing simulations contribute to employee training?
\end{itemize}

%Questions written
%Insert the questions here. 

%\bibliographystyle{plain}
%\bibliography{qanda}



%outro written
\maketitle

\section*{Introduction}
We're going to do another question and answer session today. This time I'd like to focus on the second half of the course but I am happy to take questions from any section of the course, or from any of the recommended readings. 

If you are short of questions, then I suggest: 

\begin{itemize} 
\item looking through previous slides and seeing if there are any parts you don't feel like you `get';
\item looking through past papers and seeing if there are any questions you don't feel like you could answer;
\item asking some of the starter questions.
\end{itemize} 

You can ask questions by putting up your hand or with Teams.  

Particularly interesting questions will be added to the database and may be starter questions for next year.  

\section*{Starter Questions} 
Here are some randomly generated questions for you to try asking: 

%Intro written
\begin{itemize}
  \item How can organizations use GDPR principles to build trust with customers?
  \item Should we talk about stakeholder communication in incident management? It seems important...
  \item Can I pass an audit by putting nothing on the statement of applicability?
  \item I heard some guy say that you could use NIST controls for an ISO27001 audit - is that okay?
  \item What is the difference between data controllers and data processors?
  \item How do criminal record checks differ internationally?
  \item What are the implications of the right to be forgotten in modern business?
  \item Can we learn more about Cyberwar in the context of places like the Ukraine?
  \item Why is training staff to recognize incidents a top priority?
  \item Why might attackers exploit vulnerabilities during disaster recovery?
  \item Are there any companies that shouldn't have an ISMS?
\end{itemize}

%Questions written
%Insert the questions here. 

%\bibliographystyle{plain}
%\bibliography{qanda}



%outro written
\maketitle

\section*{Introduction}
We're going to do another question and answer session today. This time I'd like to focus on the second half of the course but I am happy to take questions from any section of the course, or from any of the recommended readings. 

If you are short of questions, then I suggest: 

\begin{itemize} 
\item looking through previous slides and seeing if there are any parts you don't feel like you `get';
\item looking through past papers and seeing if there are any questions you don't feel like you could answer;
\item asking some of the starter questions.
\end{itemize} 

You can ask questions by putting up your hand or with Teams.  

Particularly interesting questions will be added to the database and may be starter questions for next year.  

\section*{Starter Questions} 
Here are some randomly generated questions for you to try asking: 

%Intro written
\begin{itemize}
  \item Surely many policies are irrelevant if people have work email on their phones?
  \item Should we talk about stakeholder communication in incident management? It seems important...
  \item How do you put a new security policy in place at a large organisation - can you just email and say `we're doing it this way now?'
  \item How does pseudonymization differ from anonymization under GDPR?
  \item Have you ever been scammed?
  \item What is the role of risk assessment in disaster recovery planning?
  \item Given that equipment is really cheap compared to the cost of employees, why is the return of assets important when an employee leaves?
  \item How do systemic biases affect diversity in cybersecurity careers?
  \item Tiger kidnappings seem really interesting - do you have any thoughts?
  \item Can we learn more about Cyberwar in the context of places like the Ukraine?
  \item Is a statement of applicability public?
\end{itemize}

%Questions written
%Insert the questions here. 

%\bibliographystyle{plain}
%\bibliography{qanda}



%outro written
\maketitle

\section*{Introduction}
We're going to do another question and answer session today. This time I'd like to focus on the second half of the course but I am happy to take questions from any section of the course, or from any of the recommended readings. 

If you are short of questions, then I suggest: 

\begin{itemize} 
\item looking through previous slides and seeing if there are any parts you don't feel like you `get';
\item looking through past papers and seeing if there are any questions you don't feel like you could answer;
\item asking some of the starter questions.
\end{itemize} 

You can ask questions by putting up your hand or with Teams.  

Particularly interesting questions will be added to the database and may be starter questions for next year.  

\section*{Starter Questions} 
Here are some randomly generated questions for you to try asking: 

%Intro written
\begin{itemize}
  \item What are the implications of the right to be forgotten in modern business?
  \item Can we learn more about Cyberwar in the context of places like the Ukraine?
  \item How do criminal record checks differ internationally?
  \item How do you put a new security policy in place at a large organisation - can you just email and say `we're doing it this way now?'
  \item What's the difference between Risk Management and Enterprise Risk Management?
  \item How does security management work for a computer games company?
  \item How do organizations determine the scope of their disaster recovery plans?
  \item Is `Risk Appetite' the same as 'Risk Culture'
  \item How do organizations balance trust with thorough screening processes?
  \item What is the role of risk assessment in disaster recovery planning?
  \item Why might attackers exploit vulnerabilities during disaster recovery?
\end{itemize}

%Questions written
%Insert the questions here. 

%\bibliographystyle{plain}
%\bibliography{qanda}



%outro written
\maketitle

\section*{Introduction}
We're going to do another question and answer session today. This time I'd like to focus on the second half of the course but I am happy to take questions from any section of the course, or from any of the recommended readings. 

If you are short of questions, then I suggest: 

\begin{itemize} 
\item looking through previous slides and seeing if there are any parts you don't feel like you `get';
\item looking through past papers and seeing if there are any questions you don't feel like you could answer;
\item asking some of the starter questions.
\end{itemize} 

You can ask questions by putting up your hand or with Teams.  

Particularly interesting questions will be added to the database and may be starter questions for next year.  

\section*{Starter Questions} 
Here are some randomly generated questions for you to try asking: 

%Intro written
\begin{itemize}
  \item How do criminal record checks differ internationally?
  \item How do corrective actions during incident management differ from risk treatments?
  \item Have you ever been scammed?
  \item Given that equipment is really cheap compared to the cost of employees, why is the return of assets important when an employee leaves?
  \item How do organizations determine the scope of their disaster recovery plans?
  \item Is `Risk Appetite' the same as 'Risk Culture'
  \item Can I pass an audit by putting nothing on the statement of applicability?
  \item How do phishing simulations contribute to employee training?
  \item How does pseudonymization differ from anonymization under GDPR?
  \item How does security management work for a computer games company?
  \item How do you put a new security policy in place at a large organisation - can you just email and say `we're doing it this way now?'
\end{itemize}

%Questions written
%Insert the questions here. 

%\bibliographystyle{plain}
%\bibliography{qanda}



%outro written
\maketitle

\section*{Introduction}
We're going to do another question and answer session today. This time I'd like to focus on the second half of the course but I am happy to take questions from any section of the course, or from any of the recommended readings. 

If you are short of questions, then I suggest: 

\begin{itemize} 
\item looking through previous slides and seeing if there are any parts you don't feel like you `get';
\item looking through past papers and seeing if there are any questions you don't feel like you could answer;
\item asking some of the starter questions.
\end{itemize} 

You can ask questions by putting up your hand or with Teams.  

Particularly interesting questions will be added to the database and may be starter questions for next year.  

\section*{Starter Questions} 
Here are some randomly generated questions for you to try asking: 

%Intro written
\begin{itemize}
  \item How does security management work for a military unit?
  \item Surely many policies are irrelevant if people have work email on their phones?
  \item Why might attackers exploit vulnerabilities during disaster recovery?
  \item Are there any companies that shouldn't have an ISMS?
  \item What is the role of risk assessment in disaster recovery planning?
  \item How do organizations determine the scope of their disaster recovery plans?
  \item What are the implications of the right to be forgotten in modern business?
  \item What's the most important job of an Incident Response Team (IRT)?
  \item How does security management work for a computer games company?
  \item Given that equipment is really cheap compared to the cost of employees, why is the return of assets important when an employee leaves?
  \item How do you put a new security policy in place at a large organisation - can you just email and say `we're doing it this way now?'
\end{itemize}

%Questions written
%Insert the questions here. 

%\bibliographystyle{plain}
%\bibliography{qanda}



%outro written
\maketitle

\section*{Introduction}
We're going to do another question and answer session today. This time I'd like to focus on the second half of the course but I am happy to take questions from any section of the course, or from any of the recommended readings. 

If you are short of questions, then I suggest: 

\begin{itemize} 
\item looking through previous slides and seeing if there are any parts you don't feel like you `get';
\item looking through past papers and seeing if there are any questions you don't feel like you could answer;
\item asking some of the starter questions.
\end{itemize} 

You can ask questions by putting up your hand or with Teams.  

Particularly interesting questions will be added to the database and may be starter questions for next year.  

\section*{Starter Questions} 
Here are some randomly generated questions for you to try asking: 

%Intro written
\begin{itemize}
  \item Why might attackers exploit vulnerabilities during disaster recovery?
  \item How do criminal record checks differ internationally?
  \item How do phishing simulations contribute to employee training?
  \item Why is training staff to recognize incidents a top priority?
  \item I heard some guy say that you could use NIST controls for an ISO27001 audit - is that okay?
  \item What's the most important job of an Incident Response Team (IRT)?
  \item Should we talk about stakeholder communication in incident management? It seems important...
  \item Have you ever been scammed?
  \item Surely many policies are irrelevant if people have work email on their phones?
  \item It feels like all of the ransomware stuff is solved by just 'having a backup' do so many companies really just not back stuff up?
  \item What are the implications of the right to be forgotten in modern business?
\end{itemize}

%Questions written
%Insert the questions here. 

%\bibliographystyle{plain}
%\bibliography{qanda}



%outro written
\maketitle

\section*{Introduction}
We're going to do another question and answer session today. This time I'd like to focus on the second half of the course but I am happy to take questions from any section of the course, or from any of the recommended readings. 

If you are short of questions, then I suggest: 

\begin{itemize} 
\item looking through previous slides and seeing if there are any parts you don't feel like you `get';
\item looking through past papers and seeing if there are any questions you don't feel like you could answer;
\item asking some of the starter questions.
\end{itemize} 

You can ask questions by putting up your hand or with Teams.  

Particularly interesting questions will be added to the database and may be starter questions for next year.  

\section*{Starter Questions} 
Here are some randomly generated questions for you to try asking: 

%Intro written
\begin{itemize}
  \item How do systemic biases affect diversity in cybersecurity careers?
  \item Given that equipment is really cheap compared to the cost of employees, why is the return of assets important when an employee leaves?
  \item How can data breaches alter public perceptions of organizational ethics?
  \item Tiger kidnappings seem really interesting - do you have any thoughts?
  \item Why might attackers exploit vulnerabilities during disaster recovery?
  \item I read that the Risk Register should be confidential - doesn't that make it useless?
  \item Should we talk about stakeholder communication in incident management? It seems important...
  \item How do phishing simulations contribute to employee training?
  \item What is the difference between data controllers and data processors?
  \item At what point in a security incident should you tell the police??
  \item How do you put a new security policy in place at a large organisation - can you just email and say `we're doing it this way now?'
\end{itemize}

%Questions written
%Insert the questions here. 

%\bibliographystyle{plain}
%\bibliography{qanda}



%outro written
\maketitle

\section*{Introduction}
We're going to do another question and answer session today. This time I'd like to focus on the second half of the course but I am happy to take questions from any section of the course, or from any of the recommended readings. 

If you are short of questions, then I suggest: 

\begin{itemize} 
\item looking through previous slides and seeing if there are any parts you don't feel like you `get';
\item looking through past papers and seeing if there are any questions you don't feel like you could answer;
\item asking some of the starter questions.
\end{itemize} 

You can ask questions by putting up your hand or with Teams.  

Particularly interesting questions will be added to the database and may be starter questions for next year.  

\section*{Starter Questions} 
Here are some randomly generated questions for you to try asking: 

%Intro written
\begin{itemize}
  \item How do systemic biases affect diversity in cybersecurity careers?
  \item How do phishing simulations contribute to employee training?
  \item What's the difference between Risk Management and Enterprise Risk Management?
  \item Say that I write a security policy and an employee ignores it and there is a breach - can I be blamed?
  \item What is the difference between data controllers and data processors?
  \item How does security management work for a military unit?
  \item Is `Risk Appetite' the same as 'Risk Culture'
  \item Why might attackers exploit vulnerabilities during disaster recovery?
  \item How can data breaches alter public perceptions of organizational ethics?
  \item How do organizations balance trust with thorough screening processes?
  \item Why is training staff to recognize incidents a top priority?
\end{itemize}

%Questions written
%Insert the questions here. 

%\bibliographystyle{plain}
%\bibliography{qanda}



%outro written
\maketitle

\section*{Introduction}
We're going to do another question and answer session today. This time I'd like to focus on the second half of the course but I am happy to take questions from any section of the course, or from any of the recommended readings. 

If you are short of questions, then I suggest: 

\begin{itemize} 
\item looking through previous slides and seeing if there are any parts you don't feel like you `get';
\item looking through past papers and seeing if there are any questions you don't feel like you could answer;
\item asking some of the starter questions.
\end{itemize} 

You can ask questions by putting up your hand or with Teams.  

Particularly interesting questions will be added to the database and may be starter questions for next year.  

\section*{Starter Questions} 
Here are some randomly generated questions for you to try asking: 

%Intro written
\begin{itemize}
  \item How do phishing simulations contribute to employee training?
  \item I read that the Risk Register should be confidential - doesn't that make it useless?
  \item How do organizations balance trust with thorough screening processes?
  \item How does security management work for a computer games company?
  \item How do corrective actions during incident management differ from risk treatments?
  \item Why do we separate into the five phases of incident management?
  \item Are there any companies that shouldn't have an ISMS?
  \item Why is training staff to recognize incidents a top priority?
  \item How can data breaches alter public perceptions of organizational ethics?
  \item What are the implications of the right to be forgotten in modern business?
  \item It feels like all of the ransomware stuff is solved by just 'having a backup' do so many companies really just not back stuff up?
\end{itemize}

%Questions written
%Insert the questions here. 

%\bibliographystyle{plain}
%\bibliography{qanda}



%outro written
\maketitle

\section*{Introduction}
We're going to do another question and answer session today. This time I'd like to focus on the second half of the course but I am happy to take questions from any section of the course, or from any of the recommended readings. 

If you are short of questions, then I suggest: 

\begin{itemize} 
\item looking through previous slides and seeing if there are any parts you don't feel like you `get';
\item looking through past papers and seeing if there are any questions you don't feel like you could answer;
\item asking some of the starter questions.
\end{itemize} 

You can ask questions by putting up your hand or with Teams.  

Particularly interesting questions will be added to the database and may be starter questions for next year.  

\section*{Starter Questions} 
Here are some randomly generated questions for you to try asking: 

%Intro written
\begin{itemize}
  \item What is the difference between data controllers and data processors?
  \item Why is the 'sliding scale' of spam relevance a key concern in marketing ethics?
  \item Say that I write a security policy and an employee ignores it and there is a breach - can I be blamed?
  \item What is the role of risk assessment in disaster recovery planning?
  \item Should we talk about stakeholder communication in incident management? It seems important...
  \item What's the difference between Risk Management and Enterprise Risk Management?
  \item I heard some guy say that you could use NIST controls for an ISO27001 audit - is that okay?
  \item How do phishing simulations contribute to employee training?
  \item Why is training staff to recognize incidents a top priority?
  \item How do organizations determine the scope of their disaster recovery plans?
  \item Can I pass an audit by putting nothing on the statement of applicability?
\end{itemize}

%Questions written
%Insert the questions here. 

%\bibliographystyle{plain}
%\bibliography{qanda}



%outro written
\maketitle

\section*{Introduction}
We're going to do another question and answer session today. This time I'd like to focus on the second half of the course but I am happy to take questions from any section of the course, or from any of the recommended readings. 

If you are short of questions, then I suggest: 

\begin{itemize} 
\item looking through previous slides and seeing if there are any parts you don't feel like you `get';
\item looking through past papers and seeing if there are any questions you don't feel like you could answer;
\item asking some of the starter questions.
\end{itemize} 

You can ask questions by putting up your hand or with Teams.  

Particularly interesting questions will be added to the database and may be starter questions for next year.  

\section*{Starter Questions} 
Here are some randomly generated questions for you to try asking: 

%Intro written
\begin{itemize}
  \item How do organizations determine the scope of their disaster recovery plans?
  \item Should we talk about stakeholder communication in incident management? It seems important...
  \item Is a statement of applicability public?
  \item I read that the Risk Register should be confidential - doesn't that make it useless?
  \item Why might attackers exploit vulnerabilities during disaster recovery?
  \item Is `Risk Appetite' the same as 'Risk Culture'
  \item It feels like all of the ransomware stuff is solved by just 'having a backup' do so many companies really just not back stuff up?
  \item How do systemic biases affect diversity in cybersecurity careers?
  \item Surely many policies are irrelevant if people have work email on their phones?
  \item In lots of the case studies a company calls in external consultants - is there a rule for when you do that?
  \item How do you put a new security policy in place at a large organisation - can you just email and say `we're doing it this way now?'
\end{itemize}

%Questions written
%Insert the questions here. 

%\bibliographystyle{plain}
%\bibliography{qanda}



%outro written
\maketitle

\section*{Introduction}
We're going to do another question and answer session today. This time I'd like to focus on the second half of the course but I am happy to take questions from any section of the course, or from any of the recommended readings. 

If you are short of questions, then I suggest: 

\begin{itemize} 
\item looking through previous slides and seeing if there are any parts you don't feel like you `get';
\item looking through past papers and seeing if there are any questions you don't feel like you could answer;
\item asking some of the starter questions.
\end{itemize} 

You can ask questions by putting up your hand or with Teams.  

Particularly interesting questions will be added to the database and may be starter questions for next year.  

\section*{Starter Questions} 
Here are some randomly generated questions for you to try asking: 

%Intro written
\begin{itemize}
  \item Should we talk about stakeholder communication in incident management? It seems important...
  \item What is the difference between data controllers and data processors?
  \item In lots of the case studies a company calls in external consultants - is there a rule for when you do that?
  \item What's the difference between Risk Management and Enterprise Risk Management?
  \item What is the role of risk assessment in disaster recovery planning?
  \item How do criminal record checks differ internationally?
  \item How can data breaches alter public perceptions of organizational ethics?
  \item Are there any companies that shouldn't have an ISMS?
  \item At what point in a security incident should you tell the police??
  \item What's the most important job of an Incident Response Team (IRT)?
  \item How can organizations use GDPR principles to build trust with customers?
\end{itemize}

%Questions written
%Insert the questions here. 

%\bibliographystyle{plain}
%\bibliography{qanda}



%outro written
\maketitle

\section*{Introduction}
We're going to do another question and answer session today. This time I'd like to focus on the second half of the course but I am happy to take questions from any section of the course, or from any of the recommended readings. 

If you are short of questions, then I suggest: 

\begin{itemize} 
\item looking through previous slides and seeing if there are any parts you don't feel like you `get';
\item looking through past papers and seeing if there are any questions you don't feel like you could answer;
\item asking some of the starter questions.
\end{itemize} 

You can ask questions by putting up your hand or with Teams.  

Particularly interesting questions will be added to the database and may be starter questions for next year.  

\section*{Starter Questions} 
Here are some randomly generated questions for you to try asking: 

%Intro written
\begin{itemize}
  \item Why is training staff to recognize incidents a top priority?
  \item Surely many policies are irrelevant if people have work email on their phones?
  \item What is the role of risk assessment in disaster recovery planning?
  \item How do phishing simulations contribute to employee training?
  \item At what point in a security incident should you tell the police??
  \item Say that I write a security policy and an employee ignores it and there is a breach - can I be blamed?
  \item Can I pass an audit by putting nothing on the statement of applicability?
  \item How does pseudonymization differ from anonymization under GDPR?
  \item How do systemic biases affect diversity in cybersecurity careers?
  \item Given that equipment is really cheap compared to the cost of employees, why is the return of assets important when an employee leaves?
  \item How do criminal record checks differ internationally?
\end{itemize}

%Questions written
%Insert the questions here. 

%\bibliographystyle{plain}
%\bibliography{qanda}



%outro written
\maketitle

\section*{Introduction}
We're going to do another question and answer session today. This time I'd like to focus on the second half of the course but I am happy to take questions from any section of the course, or from any of the recommended readings. 

If you are short of questions, then I suggest: 

\begin{itemize} 
\item looking through previous slides and seeing if there are any parts you don't feel like you `get';
\item looking through past papers and seeing if there are any questions you don't feel like you could answer;
\item asking some of the starter questions.
\end{itemize} 

You can ask questions by putting up your hand or with Teams.  

Particularly interesting questions will be added to the database and may be starter questions for next year.  

\section*{Starter Questions} 
Here are some randomly generated questions for you to try asking: 

%Intro written
\begin{itemize}
  \item Why is the 'sliding scale' of spam relevance a key concern in marketing ethics?
  \item Why do we separate into the five phases of incident management?
  \item I read that the Risk Register should be confidential - doesn't that make it useless?
  \item How does pseudonymization differ from anonymization under GDPR?
  \item What is the role of risk assessment in disaster recovery planning?
  \item Say that I write a security policy and an employee ignores it and there is a breach - can I be blamed?
  \item How do systemic biases affect diversity in cybersecurity careers?
  \item Can I pass an audit by putting nothing on the statement of applicability?
  \item I heard some guy say that you could use NIST controls for an ISO27001 audit - is that okay?
  \item How do organizations determine the scope of their disaster recovery plans?
  \item How do corrective actions during incident management differ from risk treatments?
\end{itemize}

%Questions written
%Insert the questions here. 

%\bibliographystyle{plain}
%\bibliography{qanda}



%outro written
\end{document}


