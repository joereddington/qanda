\maketitle

\section*{Introduction}

Traditional lectures often follow a one-way flow of information, where the instructor presents material and students passively absorb the content. 
While this approach can be useful, it is not effective method for fostering deep understanding and repeatable skills. In contrast, a Q\&A-based lecture flips this dynamic and enables students to take an active role in their learning process.\cite{reidsema2017flipped} 

In a Q\&A-based lecture, the session is driven by the questions of the students, allowing the lecture to be more adaptive and responsive to their needs. This type of interactive learning environment has shown to be more engaging and effective, as students not only receive answers to their questions but also participate in the exploration of the topic alongside the instructor.\cite{ZhengLanqin2020TEot} 

By shifting the focus from passive listening to active participation, Q\&A-based lectures make learning more interactive, personalized, and effective. The effectiveness of such interactive models has been supported by studies that show improved learning outcomes and motivation in environments where active learning approaches, such as flipped classrooms, are used.\cite{ZhengLanqin2020TEot}

\section{Example Questions} 
Here are some randomly generated questions for you to try: 
